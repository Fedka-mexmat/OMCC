\documentclass[a4paper,12pt]{article}
\usepackage[english,russian]{babel}
\usepackage[many]{tcolorbox}
\usepackage[utf8]{inputenc}
\usepackage[T2A]{fontenc}
\usepackage{amssymb}
\usepackage[unicode, pdftex]{hyperref}


\usepackage[
  a4paper, mag=1000, includefoot,
  left=1.1cm, right=1.1cm, top=1.2cm, bottom=1.2cm, headsep=0.8cm, footskip=0.8cm
]{geometry}

\usepackage{amsmath}
\usepackage{amssymb}
\usepackage{times}
\usepackage{mathptmx}
\usepackage{graphicx}
\usepackage{mathrsfs}
\usepackage{tikz}
\usepackage{mathtools}
\newcommand{\divisible}{\mathop{\raisebox{-2pt}{\vdots}}}
\newtheorem{deff}{\textit{Определение}}
\newtheorem{teo}{\textit{Теорема}}
\newtheorem{utv}{\textit{Утверждение}}
\newtheorem{lem}{\textit{Лемма}}
\newtheorem{deff2}{\textit{Определение}}
\newtheorem{teo2}{\textit{Теорема}}
\newtheorem{utv2}{\textit{Утверждение}}
\newtheorem{lem2}{\textit{Лемма}}
\newcommand{\ee}{\equiv}

\newcommand{\pp}{\partial}
\newcommand{\FI}{\varphi}
\newcommand{\TE}{\theta}
\newcommand{\AL}{\alpha}
\newcommand{\SI}{\psi}
\newcommand{\q}{\quad}
\newcommand{\pb}{\blacktriangleright}
\newcommand{\pe}{\blacktriangleleft}
\newcommand{\Ra}{\Rightarrow}
\newcommand{\bb}[1]{\mathbb{#1}}
\newcommand{\dt}{\frac{d}{dt}}
\newcommand{\fracp}[2]{\frac{\pp #1}{\pp #2}}
\newcommand{\mr}{\mathscr}
\newcommand{\pf}{\textit{Доказательство}\\}
\newcommand{\pfe}{\textit{Доказательство окончено}\\}
\newcommand{\SL}{\sum\limits}
\newcommand{\IL}{\int\limits}
\newcommand{\os}{\left(}
\newcommand{\cs}{\right)}
\newcommand{\R}{\mathbb{R}}
\newtcolorbox{mybox}[2][]{
    colback=white,
    colframe=blue!50!black,
    title=#2,
    fonttitle=\bfseries,
    enhanced,
    breakable,
    attach boxed title to top left={yshift=-\tcboxedtitleheight/2},
    boxed title style={size=small,colback=red,colframe=blue!50!black},
    #1
}
\newtcolorbox{mybox2}[2][]{
    colback=white,
    colframe=blue!50!black,
    title=#2,
    fonttitle=\bfseries,
    enhanced,
    breakable,
    attach boxed title to top left={yshift=-\tcboxedtitleheight/2},
    boxed title style={size=small,colback=blue,colframe=blue!50!black},
    #1
}


\newtcolorbox{formbox}[2][]{
    colback=white,
    colframe=blue!50!black,
    title=#2,
    fonttitle=\bfseries,
    enhanced,
    breakable,
    attach boxed title to top left={yshift=-\tcboxedtitleheight/2},
    #1
}

\usepackage{blindtext}
\usepackage{titlesec}
\title{ОМСС...}
\author{SFS}
\date{\today}
\begin{document}
\maketitle
\section*{Билеты}
\begin{enumerate}
\item \textbf{ Основные понятия и законы классической механики}

\begin{enumerate}
\item \hyperlink{bil1_1}{Тела, масса. Взаимодействия тел. Система сил, результирующая сила. Сбалансированность и попарная уравновешенность системы сил.}
\item \hyperlink{bil1_2}{Мир событий как модель вместилища для реального мира движущихся и взаимодействующих тел, системы отсчета. Родственные системы отсчета, замена системы отсчета. Движение, актуальные конфигурации тел. Основные характеристики движений и взаимодействий. Преобразование кинематических характеристик при замене системы отсчета.}
\item \hyperlink{bil1_3}{Основные законы классической механики: закон сохранения массы, закон соотнесенности сил и конфигураций тел, закон независимости мощности работы результирующих сил от системы отсчета. Теорема о сбалансированности системы сил и следствие о попарной уравновешенности сил и моментов сил этой системы.}
\item \hyperlink{bil1_4}{Большая система активно взаимодействующих тел. Инерциальные системы отсчета. Законы инерции Ньютона. Активные силы и силы инерции, даламберово равновесие. Первый и второй законы движения Эйлера. }

\end{enumerate}
\item \textbf{Основные гипотезы механики сплошной среды. Способы описания движения сплошной среды}
\begin{enumerate}
\item \hyperlink{bil2_1}{Основные гипотезы механики сплошной среды: гипотеза сплошности (гладкости движения), гипотеза распределенности массы, гипотезы распределенности массовых и поверхностных сил; контактный характер поверхностных сил. Законы движения Коши—Эйлера в механике сплошной среды.}
\item \hyperlink{bil2_2}{Способы описания движения: материальное описание, лагранжевы способы (отсчетное и относительное описание), эйлеров способ (пространственное описание), — их эквивалентность. }
\item \hyperlink{bil2_3}{Материальные производные скалярных, векторных и тензорных механических характеристик по времени. Представление вектора ускорения и уравнения неразрывности в лагранжевой и эйлеровой формах. Кинематический смысл дивергенции поля скоростей в эйлеровом описании. Изохорические движения, несжимаемость. }
\item \hyperlink{bil2_4}{Траектории движения, линии тока. Установившееся (стационарное) движение. Вихрь поля скоростей (в эйлеровом описании), вихревые линии, вихревые поверхности. Кинематические теоремы Гельмгольца о вих-ревых трубках. Безвихревые (потенциальные) движения.}
\end{enumerate}

\item \textbf{ Теория деформаций}
\begin{enumerate}
\item \hyperlink{bil3_1}{Понятие деформации элементарных материальных частиц по Коши. Аффинор деформации, однородная деформация. Полярное разложение аффинора деформации: правый и левый тензоры растяжений (чистой деформации), тензор вращений (поворота), правые и левые главные оси деформации, главные удлинения. Примеры: жесткое движение, чистая деформация.}
\item \hyperlink{bil3_2}{Кратности удлинений элементарных материальных волокон и изменение углов между ними в процессе деформации. Подходы Коши—Грина и Коши—Альманси к описанию деформаций. Меры деформаций Коши и Альманси, тензоры деформаций Грина и Альманси. Представление аффинора и тензоров деформаций в лагранжевых базисах.}
\item \hyperlink{bil3_3}{Ориентированные элементарные площадки и элементарные объемы. Деформации элементарных площадок и объемов.}
\item \hyperlink{bil3_4}{Тензоры дисторсий. Выражение тензоров деформаций Грина и Альманси через вектор перемещений, компонентные представления. }
\item \hyperlink{bil3_5}{Аддитивные тензоры растяжений и поворота. Случаи малых деформаций, малых дисторсий, классический случай «малых деформаций» (малые дисторсии и перемещения). Линейный тензор деформаций Коши. }
\item \hyperlink{bil3_6}{Относительное удлинение материального волокна, (угловой) сдвиг двух материальных волокон, относительное изменение объема в случаях малых деформаций и малых дисторсий (а также в классическом случае малых деформаций). Кинематический смысл декартовых компонент линейного тензора деформаций Коши.}
\item \hyperlink{bil3_7}{Замена отсчетной конфигурации. Актуальная конфигурация в качестве новой отсчетной (относительное описание). Тензоры скоростей дисторсий, скоростей деформаций и скоростей вращений (спин), их связь с тензорами дисторсий, деформаций и вращений относительного описания. }
\item \hyperlink{bil3_8}{Кинематический смысл спина и тензора скоростей деформаций, скорость относительного удлинения волокна, скорость (углового) сдвига двух волокон, скорость относительного изменения объема. }
\item \hyperlink{bil3_9}{ Связь тензоров скоростей деформаций и скоростей вращений с тензорами растяжений и поворота отсчетного (лагранжева) описания. Аналогия теории скоростей деформаций и классического случая малых деформаций. }

\end{enumerate}

\item \textbf{Теория напряжений. Уравнения баланса (локальная форма). Основные системы соотношений начальнокраевых задач}


\begin{enumerate}

\item \hyperlink{bil4_1}{Напряженное состояние среды. Постулат Коши. Основная лемма о попарной уравновешенности контактных усилий, следствия для массовых и контактных сил и их моментов. Фундаментальная теорема Коши о существовании тензора напряжений. }
\item \hyperlink{bil4_2}{ Тензор истинных напряжений Коши. Нормальные и касательные напряжения, смысл декартовых компонент тензора напряжений. Теорема взаимности Коши, свойство парности касательных напряжений (декартовых компонент напряжений). Главные оси напряжений, главные напряжения. Пример напряженного состояния при одноосном растяжении.}
\item \hyperlink{bil4_3}{Тензоры условных напряжений Пиолы—Кирхгофа первого и второго рода, «энергетический» тензор напряжений Ильюшина. Лагранжево и смешанное описание напряженного состояния (вектора напряжений). Связь компонент тензоров напряжений в лагранжевых (в отстчетной и актуальной конфигурациях) и смешанном базисах. Связь между различными тензорами напряжений в случаях малых деформаций и малых дисторсий. }
\item \hyperlink{bil4_4}{Общее уравнение баланса и общее уравнение поля в механике сплошной среды. Пример: баланс удельного объема и уравнение неразрывности. Результирующие массовые силы и их моменты как векторные меры на телах большой системы и их подтелах. Учет внутренних массовых сил в активных взаимодействиях тел большой системы (и их частей). Независимость суммарной плотности (внешних и внутренних) активных массовых сил от выбора тела данной большой системы.}
\item \hyperlink{bil4_5}{Баланс количества движения и первое уравнение движения Коши. Баланс момента количества движения и второе уравнение движения Коши (симметричность тензора напряжений). Представление уравнений движения через тензоры условных напряжений в лагранжевом описании.}
\item \hyperlink{bil4_6}{Граничные и начальные условия. Основная система соотношений (начально-) краевой задачи механики сплошной среды в лагранжевом описании и в классическом случае “малых деформаций”. Динамика, квазистатика, статика, необходимые условия статического равновесия. Основная система соотношений (начально-) краевой задачи в эйлеровом описании. Динамика, квазистатика, стационарные движения.}

\end{enumerate}

\item \textbf{ Основы теории определяющих соотношений}
\begin{enumerate}

\item \hyperlink{bil5_1}{Внешние воздействия и динамические процессы в телах. Преобразование компонент динамического процесса при замене системы отсчета. Понятия механических свойств сопротивления тел деформированию и определяющих соотношений. }
\item \hyperlink{bil5_2}{Основные принципы общей теории определяющих соотношений механики сплошной среды: упрощающие предположения о внутренних массовых взаимодействиях; предыстория движения, принцип детерминизма и причинности; принцип локальности; принцип материальной независимости от системы отсчета. Гипотеза макроскопической определимости механических свойств материалов, простые материалы. Рамки классической механики сплошной среды.}
\item \hyperlink{bil5_3}{Совместные следствия гипотезы макроскопической определимости и основных принципов теории определяющих соотношений. Общие приведенные формы определяющих соотношений классической механики сплошной среды А.А.Ильюшина и У.Нолла, их эквивалентность.}
\item \hyperlink{bil5_4}{Материалы с простейшими (простыми мгновенными) внутренними кинематическими связями: принцип детерминизма и определяющие соотношения. Примеры: несжимаемость, нерастяжимость, абсолютная твердость.}
\end{enumerate}


\item \textbf{  Простейшие среды и задачи}
\begin{enumerate}
\item \hyperlink{bil6_1}{Простейшие жидкости: текучесть, изотропия. Соответствие определяющим соотношениям У.Нолла и А.А.Ильюшина. Сжимаемость и несжимаемость. Простейшие жидкости с линейными определяющими соотношениями: эйлерова (идеальная) жидкость, ньютонова (линейно вязкая) жидкость. Гидростатика, непроявление вязкости.}
\item \hyperlink{bil6_2}{Упругое тело. Гиперупругость, изотропия, линейность определяющей функции. Несжимаемость. Основные предположения классической теории упругости. Закон Гука. Аналогия определяющих соотношений ньютоновой жидкости и классического изотропного линейно упругого тела.}
\item \hyperlink{bil6_3}{Задачи гидромеханики идеальных жидкостей. Уравнения Эйлера. Случай несжимаемых однородных жидкостей. }
\item \hyperlink{bil6_4}{Задачи гидромеханики линейно-вязких жидкостей. Уравнения Навье—Стокса. Гидростатика: совпадение с поведением идеальной жидкости. }
\item \hyperlink{bil6_5}{Задачи классической линейной теории упругости. Уравнения Ламе. }

\end{enumerate}
\end{enumerate}


\newpage
\begin{mybox}{\hypertarget{bil1_1}{Билет 1.1}}
\end{mybox}
\newpage
\begin{mybox}{\hypertarget{bil1_2}{Билет 1.2}}
\end{mybox}
\newpage
\begin{mybox}{\hypertarget{bil1_3}{Билет 1.3}}
\end{mybox}
\newpage
\begin{mybox}{\hypertarget{bil1_4}{Билет 1.4}}
\end{mybox}

\newpage
\begin{mybox}{\hypertarget{bil2_1}{Билет 2.1}}
\centering{ Основные гипотезы механики сплошной среды: гипотеза сплошности (гладкости движения), гипотеза распределенности массы, гипотезы распределенности массовых и поверхностных сил; контактный характер поверхностных сил. Законы движения Коши—Эйлера в механике сплошной среды.
}
\begin{enumerate}
    \item Гипотеза сплошности\\
    В любой системе отсчета $\Phi$, в любом движении $\bar{\chi}$ для любого тела $\mathscr{B}$ любые две актуальных конфигурации ($t$-конфигурация и $t'$-конфигурация) $\bar{\chi}(\mathscr{B}, t)$ и  $\bar{\chi}(\mathscr{B}, t')$ связаны отображением, диффеоморфным по отношению к материальным внутренностям этих конфигураций, а именно: пусть $\bar{x} = \bar{\chi}(b, t),\;\bar{x}' = \bar{\chi}(b, t'), \;\Omega_t = \bar{\chi}(\mathscr{B}, t), \;\Omega_{t'} = \bar{\chi}(\mathscr{B}, t') $. Тогда $\bar{x}' = \bar{\chi}(b, t') = \bar{\chi}(\bar{\chi}^{-1}(\bar{x}, t), t') = \bar{\varphi}_{t,t'}(\bar{x})\;\;\;\forall t, t'$ отображает $\Omega_t \to \Omega_{t'}$ с диффеоморфным сужением $\bar{\varphi}_{t,t'}: \bar{\chi}(\mathscr{B}, t) \to \bar{\chi}(\mathscr{B}, t')$.\\
    Замечание. Диффеоморфизм потребуется первого и второго порядков.\\
    Вывод: Таким образом, движение в МСС есть гладкая по времени смена актуальных конфигураций тел, диффеоморфная по отношению к внутренностям тел.
    \item Гипотеза неразрывности (распределенности массы)\\
    Масса является абсолютно непрерывной мерой по отношению к мере объема объекта в какой-либо одной (а в силу гипотезы сплошности и в любой другой) актуальной конфигурации тела.\\
    Теорема Родона-Никодима:\\
    Пусть $(X, F, \mu)$ -- пространство с мерой и мера $\mu\;\sigma$-конечна. Тогда если мера $\nu: F\to R$ абсолютно непрерывна относительно $\mu\;(\nu << \mu)$, то $\exists$ измеримая функция $f:X\to R: \nu(A) = \int\limits_A f(x)\mu(dx)\;\; \forall A\in F$ (интеграл Лебега).\\
    По теореме Родона-Никодима существует плотность $\rho:$\\
    $M(\mathscr{B}) = \int_{\bar{\chi}(\mathscr{B},t)} \rho(\bar{x}, t) dV = \int_{\bar{\chi}(\mathscr{B},t')} \rho(\bar{x}, t') dV$\\
    $dM = \rho(\bar{x}, t)dV_t = \rho(\bar{x}, t')dV_{t'}\Rightarrow \rho_tdV_t = \rho_{t'}dV_{t'}$\\
    $V(\bar{\chi}(\mathscr{B}, t)) = \int_{\bar{\chi}(\mathscr{B}, t)} dV,\; V(\bar{\chi}(\mathscr{B}, t')) = \int_{\bar{\chi}(\mathscr{B}, t')} dV = \int_{\bar{\chi}(\mathscr{B}, t)} J_{t, t'} dV $, где $J$ -- якобиан смены конфигурации.\\
    Следствие. $dV_{t'} = J_{t, t'}dV_t$, и потому $\rho_t dV_t = \rho_{t'} J_{t, t'} dV_{t'}\Rightarrow \rho(\bar{x}_t, t) = \rho(\bar{x}_{t'}, t') J_{t, t'}$ -- уравнение неразрывности в форме Лагранжа.
    \item Гипотеза распределенности сил.\\
    Система сил в классической МСС $\vec{f}$ есть сумма двух систем сил:
    \begin{itemize}
        \item Массовых сил ($\vec{f}_B$)
        \item Контактных (поверхтностных) сил ($\vec{f}_C$)
    \end{itemize}
    $\vec{f} = \vec{f}_B + \vec{f}_C$\\


    \begin{mybox2}{Определение}
    Массовая сила $\vec{f}_B(e, \mathscr{B}, t)$  воздействия любого фиксированного тела $\mathscr{B}$ на произвольное отделенное от него тело $e$ (в любой момент времени $t$) -- есть векторная мера $f_{B, \mathscr{B}, t}(\cdot)$, абсолютно непрерывная относительно меры массы $M(\cdot)$
    \end{mybox2}
    \begin{mybox2}{Определение}
    Компактная (поверхтностная) сила $\vec{f}_C (e, \mr{B}, t)$ воздействия для любого фиксированного тела $\mr{B}$ на произвольное отделенное от него тело $e$ (для любого момента $t$):
    \begin{enumerate}
        \item Зависит только от контактной поверхности актуальной конфигурации этих тел
        \item есть векторная мера $\vec{f}_{C, \mr{B}, t}(\cdot)$ на подмножествах контактной поверхности, абсолютно непрерывная относительно меры площади $S_{t}(\Delta \Gamma_{cont}(t))$ подмножеств $\Delta \Gamma_{cont}(t)$ общей контактной части границ ($\Delta \Gamma_{cont}(t) = \partial \bar{\chi}(\mr{B}, t) \cap \partial \bar{\chi}(e, t) $)
    \end{enumerate}
    \end{mybox2}
\end{enumerate}
\end{mybox}
\newpage
\begin{mybox}{\hypertarget{bil2_2}{Билет 2.2}}
\end{mybox}
\newpage
\begin{mybox}{\hypertarget{bil2_3}{Билет 2.3}}
\end{mybox}
\newpage
\begin{mybox}{\hypertarget{bil2_4}{Билет 2.4}}
\end{mybox}

\newpage
\begin{mybox}{\hypertarget{bil3_1}{Билет 3.1}}
\end{mybox}
\newpage
\begin{mybox}{\hypertarget{bil3_2}{Билет 3.2}}
\end{mybox}
\newpage
\begin{mybox}{\hypertarget{bil3_3}{Билет 3.3}}
\end{mybox}
\newpage
\begin{mybox}{\hypertarget{bil3_4}{Билет 3.4}}
\end{mybox}
\newpage
\begin{mybox}{\hypertarget{bil3_5}{Билет 3.5}}
\end{mybox}
\newpage
\begin{mybox}{\hypertarget{bil3_6}{Билет 3.6}}
\end{mybox}
\newpage
\begin{mybox}{\hypertarget{bil3_7}{Билет 3.7}}
\end{mybox}
\newpage
\begin{mybox}{\hypertarget{bil3_8}{Билет 3.8}}
\end{mybox}
\newpage
\begin{mybox}{\hypertarget{bil3_9}{Билет 3.9}}
\end{mybox}
\newpage
\begin{mybox}{\hypertarget{bil4_1}{Билет 4.1}}
\end{mybox}
\newpage
\begin{mybox}{\hypertarget{bil4_2}{Билет 4.2}}
\end{mybox}
\newpage
\begin{mybox}{\hypertarget{bil4_3}{Билет 4.3}}
\end{mybox}
\newpage
\begin{mybox}{\hypertarget{bil4_4}{Билет 4.4}}
\end{mybox}
\newpage
\begin{mybox}{\hypertarget{bil4_5}{Билет 4.5}}
\end{mybox}
\newpage
\begin{mybox}{\hypertarget{bil4_6}{Билет 4.6}}
\end{mybox}


\newpage
\begin{mybox}{\hypertarget{bil5_1}{Билет 5.1}}
\end{mybox}
\newpage
\begin{mybox}{\hypertarget{bil5_2}{Билет 5.2}}
\end{mybox}
\newpage
\begin{mybox}{\hypertarget{bil5_3}{Билет 5.3}}
\end{mybox}
\newpage
\begin{mybox}{\hypertarget{bil5_4}{Билет 5.4}}
\end{mybox}


\newpage
\begin{mybox}{\hypertarget{bil6_1}{Билет 6.1}}
\end{mybox}
\newpage
\begin{mybox}{\hypertarget{bil6_2}{Билет 6.2}}
\end{mybox}
\newpage
\begin{mybox}{\hypertarget{bil6_3}{Билет 6.3}}
\end{mybox}
\newpage
\begin{mybox}{\hypertarget{bil6_4}{Билет 6.4}}
\end{mybox}
\newpage
\begin{mybox}{\hypertarget{bil6_5}{Билет 6.5}}
\end{mybox}
\end{document}
